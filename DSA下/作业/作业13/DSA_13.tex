\documentclass[UTF8]{ctexart}
\usepackage{dirtree}
\usepackage{listings}
\usepackage{xcolor}
\usepackage{graphicx}
\usepackage{enumerate}
\usepackage[a4paper]{geometry}
\usepackage{amsmath,amsthm,mathtools,amsfonts}
\usepackage{diagbox}
\usepackage{multirow,makecell}
\usepackage{float}
\usepackage{url}
\usepackage[nottoc]{tocbibind}
\usepackage{float}
\usepackage[colorlinks=true, linkcolor=black]{hyperref}
\lstset{
 columns=fixed,
 numbers=left,                                        % 在左侧显示行号
 numberstyle=\tiny\color{gray},                       % 设定行号格式
 basicstyle=\small\ttfamily,
 frame=none,                                          % 不显示背景边框
 backgroundcolor=\color[RGB]{245,245,244},            % 设定背景颜色
 keywordstyle=\color[RGB]{40,40,255},                 % 设定关键字颜色
 numberstyle=\footnotesize\color{darkgray},           
 commentstyle=\color{gray}\ttfamily,                  % 设置代码注释的格式
 stringstyle=\rmfamily\slshape\color[RGB]{128,0,0},   % 设置字符串格式
 showstringspaces=false,
 breaklines=true,
 language=python
}

\newcommand{\refe}[1]{Eq.\ref{#1}}
\newcommand{\reft}[1]{Theory.\ref{#1}\ }
\newcommand{\refl}[1]{Lemma.\ref{#1}\ }
\newcommand{\reff}[1]{图\ref{#1}\ }
\newtheorem{example}{例}
\newtheorem{definition}{定义}
\newtheorem{theorem}{定理}[section]
\newtheorem{lemma}{引理}
\geometry{bottom=2cm,left=1cm,right=1cm}
\title{DSA\_13}
\author{邵宁录\space\space\space\space2018202195}
\pagestyle{plain}
\begin{document}
    \maketitle
    % \tableofcontents
    % \newpage
    \section{24.1-3}
    在每一次循环所有点之前,先把所有的$v.d$给记录下来,然后进行$RELAX$操作。
    若本轮循环结束时,$v.d$没有发生变化,那么就不需要再更新,即此次循环为第$m+1$次。

    伪代码如下:
    \begin{lstlisting}
BELLMAN-FORD(G, w, s)
    INITIALIZE-SINGLE-SOURCE(G, s)
    for i = 1 to |G.V| - 1:
        for each v in G.V:
            save v.d to T
        for each edgr(u, v) in G.E:
            RELAX(u, v, w)
        for each v in G.V:
            if v.d != T[v].d:
                goto row 11
    for each edge(u, v) in G.E:
        if v.d > u.d + w(u, v):
            return FALSE
    return TRUE
    \end{lstlisting}
    \section{24.3-8}
    由于$Dijkstra$算法的时间复杂度依赖于最小优先队列队列的实现,所以想要改变算法的时间复杂度应当从优先队列的实现角度来思考。

    想要实现$O(WV+E)$的时间复杂度,我们只需要用一个二维数组来实现优先队列。

    令$A$为一个$W\times A$的二维数组,其中,具有长度为$d$的节点在$A[d]$的$list$中。

    $EXTRACT-MIN$操作的时间复杂度为$O(W)$,因为需要找一遍长度为$d$的节点中最小的$d$。因此所有的该操作加起来的时间复杂度为$O(WV)$。

    $DECREASE-KEY$操作的时间复杂度为$O(1)$的时间,因为检查一条边只需要随机访问就能完成。因此所有的该操作的时间复杂度为$O(E)$。

    综上,总的时间复杂度为$O(WV+E)$。
    \section{25.2-7}
    \textbf{递归式如下:}
    \begin{equation*}
        \phi^{(k)}_{ij}=\left\{\begin{aligned}
            \ \ &\phi^{(k-1)}_{ij}& & & {if\ d^{(k)}_{ik}+d^{(k)}_{ij}\ge d^{(k-1)}_{ij}}\\
            &k& & & {otherwise}
        \end{aligned}
        \right.
    \end{equation*}

    \textbf{理由如下:}

    如果$d^{(k)}_{ik}+d^{(k)}_{ij}\ge d^{(k-1)}_{ij}$,说明此时在$Floyd-Warshall$算法中,应当取$d^{(k)}_{ij}=d^{(k-1)}_{ij}$。
    这说明取$k$个点与取$k-1$个点所得出的最短路径是相同的,因此$\phi^{(k)}_{ij}=\phi^{(k-1)}_{ij}$。
    否则,说明新加入的结点$k$跟${1,2,\ldots,k-1}$中的点构成了一条新的最短路径,因此结点$k$是其中最大的节点,即$\phi^{(k)}_{ij}=k$

    \textbf{修改后的代码:}
    \begin{lstlisting}
FLOYD-WARSHALL(W)
    n = W.rows
    Phi_0 = W
    for k = 1 to n:
        let Phi_k = Phi_0
        for i = 1 to n:
            for j = 1 to n:
                if d_(k)[i][k] + d_(k)[k][j] < d_(k-1)[i][j]:
                Phi_k[i][j] = k


PRINT-ALL-PAIRS-SHORTEST-PATH(Phi, i, j)
    if i == j:
        print i
    else if Phi[i][j] == NIL:
        print "no path from" i "to" j "exist"
    else:
        PRINT-ALL-PAIRS-SHORTEST-PATH(Phi, i, Phi[i][j])
        print j
    \end{lstlisting}

    \textbf{相似:}
    
    $\Phi$与矩阵链式乘法中的$s$表非常类似,因为都是用相似的递归式进行计算的。
\end{document}